%%% Do not change from BEGIN to END
%%% BEGIN
\documentclass[a4paper,10pt]{article}

\usepackage[utf8]{inputenc}
\usepackage[T1]{fontenc}
\usepackage[danish,english]{babel}
\usepackage{graphicx}
\usepackage[a4paper,margin=2.7cm]{geometry}

\usepackage[math]{kurier}
\usepackage{amsmath}
\usepackage{amssymb}
\usepackage{amsfonts}
\usepackage{enumerate}

\usepackage{tikz}
\usetikzlibrary{arrows,decorations.pathmorphing,backgrounds,positioning,fit,matrix}


\setlength{\parindent}{0pt}
\setlength{\parskip}{2ex} 


\renewcommand{\vec}[1]{\ensuremath {\mathbf #1}}

%%% Settings for the Instructor
\newcommand{\courseid}{DM559}
\newcommand{\coursename}{Linear and Integer Programming}
\newcommand{\term}{Spring 2018}
\newcommand{\dept}{Department of Mathematics and Computer Science}

%%% Settings for the Student
%%%%%%%%%%%%%%%%%%%%%%%%%%%%%%%%%%%%%%%%%%%%%%%%%%%%%%%%%%%%
%%%%%%%%%%%%%%%%%%%%%%%%%%%%%%%%%%%%%%%%%%%%%%%%%%%%%%%%%%%%
\author{WRITE-HERE-YOUR-NAME}

\makeatletter
\let\runauthor\@author
\let\runtitle\@title
\makeatother
%%%%%%%%%%%%%%%%%%%%%%%%%%%%%%%%%%%%%%%%%%%%%%%%%%%%%%%%%%%%
%%%%%%%%%%%%%%%%%%%%%%%%%%%%%%%%%%%%%%%%%%%%%%%%%%%%%%%%%%%%

%%%%%%%%%%%%%%%%%%%%%%%%%%%%%%%%%%%%%%%%%%%%%%%%%%%%%%%%%%%%

\usepackage{fancyhdr}
\usepackage{lastpage}
\usepackage{listings}
\lstset{language=Python,
  basicstyle=\ttfamily\lst@ifdisplaystyle\footnotesize\fi,%\fontfamily{pzc}\selectfont,%
  stringstyle=\ttfamily,
  commentstyle=\ttfamily,
  showstringspaces=false,
  frame=lines, 
  breaklines=true, tabsize=2,
  extendedchars=true,inputencoding=utf8
}
%\lstavoidwhitepre


\pagestyle{fancy} 
\lhead{{\courseid -- \term }} 
\chead{}
\rhead{\runauthor}
\cfoot{Page \thepage\ of \pageref{LastPage}}

\fancypagestyle{plain}{
\lhead{\dept\\
University of Southern Denmark, Odense}
\chead{}
\rhead{\today\\
\runauthor}
\lfoot{}
\rfoot{}
%\renewcommand{\headrulewidth}{0pt}
}


\title{\begin{flushleft}
\vspace{-4ex}
\courseid~-- \coursename \\[0.2cm]
{\Large Answers to Obligatory Assignment 0.1, \term \\[3ex]
\hrule}
\end{flushleft}
}


\date{}

%%% END


\begin{document}
\maketitle

%%%%%%%%%%%%%%%%%%%%%%%%%%%%%%%%%%%%%%%%%%%%%%%%%%%%%%%%%%%%
%%%%%%%%%%%%%%%%%%%%%%%%%%%%%%%%%%%%%%%%%%%%%%%%%%%%%%%%%%%%
\section*{Exercise 1}
%%%%%%%%%%%%%%%%%%%%%%%%%%%%%%%%%%%%%%%%%%%%%%%%%%%%%%%%%%%%
%%%%%%%%%%%%%%%%%%%%%%%%%%%%%%%%%%%%%%%%%%%%%%%%%%%%%%%%%%%%



Your reply to Exercise 1


Template for source code inclusion:

\begin{lstlisting}
import numpy as np

CPRN = ... # your CPR number
np.random.seed(CPRN)
np.set_printoptions(precision=3)

def generate_data():
   '''
   return the matrix A
   Example:
   >>> A = generate_data()
   '''
   pass;


d = np.random.randint(10000,100000,3)
A = generate_data() 

print A
print d
\end{lstlisting}


Your reply to Exercise 1.b


Template for figure inclusion:

\begin{figure}[htb]
\begin{center}
%% uncomment the line below and change the figure file. 
%  \includegraphics[width=1.4\textwidth]{ex1}
\end{center}
\end{figure}

%%%%%%%%%%%%%%%%%%%%%%%%%%%%%%%%%%%%%%%%%%%%%%%%%%%%%%%%%%%%
%%%%%%%%%%%%%%%%%%%%%%%%%%%%%%%%%%%%%%%%%%%%%%%%%%%%%%%%%%%%
\newpage
\section*{Exercise 2}
%%%%%%%%%%%%%%%%%%%%%%%%%%%%%%%%%%%%%%%%%%%%%%%%%%%%%%%%%%%%
%%%%%%%%%%%%%%%%%%%%%%%%%%%%%%%%%%%%%%%%%%%%%%%%%%%%%%%%%%%%



\end{document}
          



